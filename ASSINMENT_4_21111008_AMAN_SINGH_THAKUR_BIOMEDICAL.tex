\documentclass[12pt]{article}

\usepackage{graphicx}
\graphicspath{{img/}}
\usepackage{hyperref}
\hypersetup{colorlinks=true, citecolor=blue , linkcolor=blue,urlcolor=blue}

\title{NATIONAL INSTITUTE OF TECHNOLOGY\\
(NIT)RAIPUR (C.G.)}
\author {SESSION 2021-22}
\date{}

\usepackage{lipsum}
\usepackage{fancyhdr}
\pagestyle{fancy}
\lhead{ROLL NO. 21111008}
\rhead{PAGE \thepage}
\cfoot{AMAN SINGH THAKUR}
\renewcommand{\headrulewidth}{0.4pt}
\renewcommand{\footrulewidth}{0.4pt}
\begin{document}
\maketitle
\begin{figure}[h]
\centering
\includegraphics[scale=0.5]{NIT.png}
\end{figure}
\author{------------------------------ASSINMENT-4------------------------------ \\
.\> \>\> \>\> \>\>\> \>\> \>\>\>\>\>\>  \>\> \>\>\> \>\> \>\> \> \>\> \>\> \>\> \>\> ON\\ 
.\> \>\> \>\> \>\ {DISRUPTIVE INNOVATIONS IN    HEALTHCARE}\\
------------------------------------------------------------------------------------------
 }\\


\author{GUIDANCE BY \> \>\> \>\> \>\>\> \>\> \>\>\>\>\>\>  \>\> \>\>\> \>\> \>\> \> \>\> \>\> \>\> \>\>  SUBMITTED BY }\\


\author{DR.SAURABH GUPTA\>  \>\> \>\> \>  \>\> \>\>\>\>\>  \>\> \>\>\>  \>\> \>\> \>  \>\> AMAN SINGH THAKUR }\\

\author{ ASSISTANT PROFESSOR\>  \>\> \>  \> \>   \>\> \>\> \> \>\> \> \>\> \>  ROLL NO.:--- 21111008 }\\
\author{BIOMEDICAL DEPARTMENT\>  \>\> \>\> \>  \>\> \>\> \>\>   \>\> BRANCHE:--BIOMEDICAL  }\\
\author{.\>   \>\> \>\> \>  \>\> \>\> \>\> \>\> \>\> \>  \>\> \>\> \>\>\>\> \>\> \>\>  \>\> \>\> \>\> \>\>\> \>\> \>   \>\> \>\> \>\> \>\>\> \>\> \>    \>\> \>\> \>\> \>\> \>\> \>  \>\> \>\> \>\>  \>\> ENGINEERING }
\clearpage
\tableofcontents
\clearpage
\section{DISRUPTIVE INNOVATION  }
When  a new product, service, or business model is helps  to create a new market then it considered as “disruptive”, eventually disrupting existing markets and displacing previous technologies.disruptive innovation has proved to be a powerful way of thinking about innovation-driven growth.

Firstly the term “disruptive innovation” is given by Clayton M. Christensen [FIGURE \ref{fig_CR}]   in 1995. At the time, Clayton M. Christensen was a Harvard Business Professor and studying how certain innovations transform industries and why certain innovations transform industries.
\begin{figure}[h]
\centering
\includegraphics[scale=1]{CR.jpg}
\caption{Clayton M. Christensen}
\label{fig_CR}
\end{figure}
Disruptive innovation has happened again and again, in industries like computing, photography, telecommunications, and retail. Studies of these industries reveal two important facts about disruptive innovation:\\


     \>(i) The change in disruptive innovation has always been positive, it give us better products and services than before.\\
     
(ii) It is usually new entrants into the market that figure out a better way of doing things.\\

Of course, disruptive innovation, like any transformation, has winners and losers. Powerful institutional forces fight simpler alternatives because innovations challenge their livelihoods.
\subsection{Disruptive technology in healthcare}
In the early 2000s, the healthcare industry was ripe for disruptive innovation. As Christensen and colleagues wrote article in the Harvard Business Review, “Health care delivery is convoluted, expensive, and often deeply dissatisfying to consumers.”

The authors said that a whole host of disruptive innovations could end the crisis – but only if entrenched powers got out of the way and let market forces play out.

“If the natural process of disruption is allowed to proceed,” they wrote, “we’ll be able to build a new system that’s characterized by lower costs, higher quality, and greater convenience than could ever be achieved under the old system.”

As predicted, we have seen dramatic changes due to disruptive technology in healthcare. Providers have adopted electronic health records, patients can view their records via online portals, and smartphones have changed the ways we access, deliver, and document care.

Still, it’s best to think of disruptive innovation as an ongoing process. The nature of technology means that new innovations are always shaking things up.

\subsection{What is the current state of disruption?}
Christensen and colleagues wrote in the early 2000s, In the healthcare industry, entrepreneurs and newcomers frequently cause  disruption .  recently a company surveyed healthcare executives, clinical leaders, and clinicians, who concluded that disruptive innovation in healthcare will come from non traditional organizations. “Whether talking about disruptive innovation in hospitals, health care ,IT or primary care” .

Startups offer higher promises of disruptive innovation we know that from review of 400 digital technology projects and collaborations . Christensen write, “Established companies rely more on step-by-step innovation to support their current business models, while start-ups are flexible..”

While targeted start-ups will bring innovation, those on the ground agree that hospitals and health systems are the sectors with the greatest need for disruption. 
\clearpage
\subsection{Examples of disruptive healthcare technology}
Disruption innovations is happening everywhere in healthcare – from AI to  3D printing and robotics. Here are a few  examples of disruptive innovations healthcare technology:
\subsubsection{Consumer devices, wearables and apps }
Consumer devices, wearables and apps – In the past, patients could only get biometric data when they went to the doctor’s office. Now health data gathered from smartwatches [FIGURE \ref{fig_C}] and mobile fitness trackers allow consumers to play a new role in their health journey.
\begin{figure}[h]
\centering
\includegraphics[scale=0.5]{C.jpg}
\caption{Equipment that  Providing Biofeedback, Monitoring Stress.}
\label{fig_C}
\end{figure}
\subsubsection{AI and machine learning }
Artificial intelligence (AI) and machine learning – Artificial intelligence will definitely be one of the most transformational technologies and boosters for human society in the twenty-first century, which is only two decades old (AI). The concept that AI and related services and platforms would transform global productivity, working patterns, and lifestyles, as well as create massive wealth, is well-established.  AI applications are everywhere in healthcare, from patient intake to predictive analytics to new drug develompment. This technology [FIGURE \ref{fig_A}]  is changing how health systems operate and how care is delivered.
\begin{figure}[h]
\centering
\includegraphics[scale=0.6]{A.jpg}
\caption{Artificial Intelligence in Healthcare}
\label{fig_A}
\end{figure}
\clearpage
\subsubsection{Telemedicine  }
Telemedicine – Telemedicine, also known as telehealth or e-medicine, is the remote delivery of healthcare services over telecommunications infrastructure, such as chat and consultations. Telemedicine allows doctors to evaluate, diagnose, and treat patients without having to see them in person. Patients can communicate with doctors from the comfort of their own homes using personal technology or by visiting a telehealth booth.COVID-19 accelerated the expansion of telemdicine, but with long-ranging impacts. Most patients say they are interested in virtual care  [FIGURE \ref{fig_T}]  going forward.
\begin{figure}[h]
\centering
\includegraphics[scale=0.7]{T.png}
\caption{Future of Telemedicine in Healthcare.}
\label{fig_T}
\end{figure}
\clearpage
\subsubsection{Blockchain  } Blockchain - is a database technology that stores data in a way that enhances security and usability. Blockchain has the potential to build a single way of storing and collecting health records in a secure and timely manner by authorised users. Uncountable mistakes can be avoided, faster diagnoses and actions are possible, and care can be customised to each patient by eliminating miscommunication between different healthcare personnel involved in caring for the same patient.This innovation is changing many aspects of healthcare, including patient records, supply and distribution, and research.
These are just a few examples of technologies  [FIGURE \ref{fig_B}]  creating new markets and changing the way healthcare happens.
\begin{figure}[h]
\centering
\includegraphics[scale=0.3]{B.png}
\caption{ Blockchain in healthcare }
\label{fig_B}
\end{figure}
\clearpage
\subsection{How disruptive innovation happens}
In addition to technological innovation, experts say there are two processes that often facilitate disruption in healthcare specifically.

\subsubsection{Decentralization} 

The first is decentralization. Disruption typically involves new market entrants creating products that bring in new consumers. But in healthcare, everyone is already a consumer. In healthcare, disruption often shifts care from hospitals to clinics and office settings, and even into patients’ homes. Telemedicine is the most obvious example.

\subsubsection{Transference of skills} 

The second process is matching doctor's skill level with the difficulty of the medical problem. This means transferring skills [FIGURE \ref{fig_TS}]  from highly trained, expensive personnel, to more affordable providers, including technology-based care. This helps address expensive care due to overshoot of patient needs by health care institutions.

Christensen says,disruptive innovation is necessary,  both for the healthcare industry and to improve patient care. But it’s only possible in the right environment. .Christensen and colleagues conclude, “health care regulators need to ask how they can enable disruptive innovations to emerge.” Rather of working to preserve the existing system.
\begin{figure}[h]
\centering
\includegraphics[scale=0.3]{TS.png}
\caption{Transference of skills.}
\label{fig_TS}
\end{figure}
\clearpage
\end{document}