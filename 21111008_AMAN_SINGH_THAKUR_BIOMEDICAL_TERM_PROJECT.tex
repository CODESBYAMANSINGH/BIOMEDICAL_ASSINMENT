\documentclass[12pt]{article}

\usepackage{graphicx}
\graphicspath{{img/}}
\usepackage{hyperref}
\hypersetup{colorlinks=true, citecolor=blue , linkcolor=blue,urlcolor=blue}

\title{NATIONAL INSTITUTE OF TECHNOLOGY\\
(NIT)RAIPUR (C.G.)}
\author {SESSION 2021-22}
\date{}

\usepackage{lipsum}
\usepackage{fancyhdr}
\pagestyle{fancy}
\lhead{ROLL NO. 21111008}
\rhead{PAGE \thepage}
\cfoot{AMAN SINGH THAKUR}
\renewcommand{\headrulewidth}{0.4pt}
\renewcommand{\footrulewidth}{0.4pt}
\begin{document}
\maketitle
\begin{figure}[h]
\centering
\includegraphics[scale=0.5]{NIT.png}
\end{figure}
\author{------------------------------TERM PROJECT------------------------------ \\
.\> \>\> \>\> \>\>\> \>\> \>\>\>\>\>\>  \>\> \>\>\> \>\> \>\> \> \>\> \>\> \>\> \>\> ON\\ 
{BIOMETRIC APPROACH TO FILTER MICRO PLASTIC FROM SEA WATER }\\
------------------------------------------------------------------------------------------
 }\\


\author{GUIDANCE BY \> \>\> \>\> \>\>\> \>\> \>\>\>\>\>\>  \>\> \>\>\> \>\> \>\> \> \>\> \>\> \>\> \>\>  SUBMITTED BY }\\


\author{DR.SAURABH GUPTA\>  \>\> \>\> \>  \>\> \>\>\>\>\>  \>\> \>\>\>  \>\> \>\> \>  \>\> AMAN SINGH THAKUR }\\

\author{ ASSISTANT PROFESSOR\>  \>\> \>  \> \>   \>\> \>\> \> \>\> \> \>\> \>  ROLL NO.:--- 21111008 }\\
\author{BIOMEDICAL DEPARTMENT\>  \>\> \>\> \>  \>\> \>\> \>\>   \>\> BRANCHE:--BIOMEDICAL  }\\
\author{.\>   \>\> \>\> \>  \>\> \>\> \>\> \>\> \>\> \>  \>\> \>\> \>\>\>\> \>\> \>\>  \>\> \>\> \>\> \>\>\> \>\> \>   \>\> \>\> \>\> \>\>\> \>\> \>    \>\> \>\> \>\> \>\> \>\> \>  \>\> \>\> \>\>  \>\> ENGINEERING }
\clearpage
\tableofcontents
\clearpage
\section{Acknowledgment}

I would like to extend my sincere thanks to my teacher\\
DR.SAURABH GUPTA Sir, who has helped me in 
this TERM PROJECT and has
 always been very cooperative, gave me valuable suggestions and 
 guidelines during the completion of the project 
  “ BIOMETRIC APPROACH TO FILTER MICRO PLASTIC FROM SEA WATER ” .

.

.

.

.

.

.


\> \>\> \>\> \>\>\> \>\> \>\>\>\>\>\>  \>\> \>\>\> \>\> \>\> \> \>\> \>\> \>\> \>\> \>\> \>\> \>\> \>\> \> \>\> \>\> \>\> \>\> \>\> \>\> AMAN SINGH THAKUR

\> \>\> \>\> \>\>\> \>\> \>\>\>\>\>\>  \>\> \>\>\>\> \>\> \>\> \>\>\> \>\> \>\>\>\>\>\>  \>\> \>\> \> \>\> \>\> \>\> \>\> \>\> \>\> \>\> \>\> \> \>\> \>\> \>\> \>\> \>\> \>\> \>\> \>\> \>\> \>\> 21111008 


\> \>\> \>\> \>\>\> \>\> \>\>\>\>\>\>  \>\> \>\>\>\> \>\> \>\> \>\>\> \>\> \>\>\>\>\>\>  \>\> \>\> \> \>\> \>\> \>\> \>\> \>\> \>\> \>\> \>\> \> \>\> \>\> \>\>  First Semester, 

\> \>\> \>\> \>\>\> \>\> \>\>\>\>\>\>  \>\> \>\>\>\> \>\> \>\> \>\>\> \>\> \>\>\>\>\>\>  \>\> \>\> \> \>\> \>\> \>\> \>\> \>\> \>\> \>\>  Biomedical Engineering
 
\> \>\> \>\> \>\>\> \>\> \>\>\>\>\>\>  \>\> \>\>\>\> \>\> \>\> \>\>\> \>\> \>\>\>\>\>\>\>\>\>\>  \>\> \>\> \> \>\>  National Institute of Technology,

 
\> \>\> \>\> \>\>\> \>\> \>\>\>\>\>\>  \>\> \>\>\>\> \>\> \>\> \>\>\> \>\> \>\>\>\>\>\>  \>\> \>\> \> \>\> \>\> \>\> \>\> \>\> \>\> \>\> \>\> \>\> \>\> \>\> \>\> \>\>  Raipur 

.

.

.

.

.

.

Date of Submission: 
04/04/2022

\clearpage
\section{Abstract}
Microplastics are a global environmental concern that demands the development of innovative methods for removing them from water, soil, and the air. Scientific studies and engineering are focusing on new materials in combination with simple methods for microplastic removal, specifically for use in water resources. The effects of water concentration and temperature on the agglomeration-fixation reaction of microplastics with organosilanes was investigated in our case study. At temperatures ranging from 7.5 to 40 degrees Celsius, we compared biologically treated municipal wastewater, saltwater, and desalinated water.
\begin{figure}[h]
\centering
\includegraphics[scale=0.8]{MP.jpg}
\caption{Microplastic removal .}
\label{fig_ROB}
\end{figure}
 Microplastic removal was unaffected by temperature changes or the water compositions studied. ICP-OES and DOC measurements were used to monitor the organosilane residues that remained in the water following the fixing procedure.
In the waterways, only one of the organosilanes tested had no dissolved residues. Microplastic refers to the variety of polymers with varying characteristics and surface chemistry. As a result, we examined the process efficiency for polyethylene, polypropylene, polyamide, polyester, and polyvinyl chloride, which are all typical polymer types with diverse chemical compositions. Microplastics and organosilanes' polarity had a significant impact on removal efficiency. Organosilanes' organic groups can be chemically modified to adapt to various polymer kinds.

\clearpage

\section{Introduction}

Micro plastics are one of today's most difficult environmental issues. Mismanaged plastic garbage has been entering the environment since the beginnings of mass manufacture in 1950 . The plastic becomes brittle over time, fragmenting into smaller and smaller plastic bits. Micro plastics are described as plastic pieces with a diameter of less than 5 mm. Micro plastics can also be discharged directly into the environment, such as textile fibres released during washing or tyre wear . Micro plastics decompose slowly and can spread across long distances due to their high persistence. Micro plastics are now detected in all aspects of the aquatic and terrestrial environments, as well as in the atmosphere .
\begin{figure}[h]
\centering
\includegraphics[scale=0.5]{GB.jpg}
\caption{Plastics and Microplastic garbage .}
\label{fig_ROB}
\end{figure}
The environment will become progressively polluted as a result of rising plastic usage and hence release into the environment. This puts the environment and ecosystems, as well as human health, at danger. Municipal and industrial waste water treatment plants have been identified as key point sources for micro plastics in the environment in a number of studies. Despite the removal of 95 \% to \> 99\% of micro plastics from inflowing waste water to effluent in tertiary waste water treatment facilities, the level of micro plastic contamination in the in fluent is so high that the effluent is still classified as severely polluted. This is especially notable since environmental monitoring of micro plastic pollution reveals that pollution levels after waste water treatment plant effluents are greater.
\begin{figure}[h]
\centering
\includegraphics[scale=0.15]{GB1.jpg}
\caption{Plastics and Microplastic garbage .}
\label{fig_ROB}
\end{figure}
Improved micro plastic removal throughout the wastewater treatment process is required to prevent discharge into the environment. There are also a number of micro plastic-sensitive salt water treatment techniques. Membrane-based salt water desalination is one example, where micro plastics can cause membrane fouling. Micro plastics can be transported from the environment to human meals through nearby food chain operations such sea salt manufacturing, posing a health concern. There is a need for a technology to remove micro plastics from water that is cost efficient and requires little technical effort to assure the operation of micro plastic sensitive water utilising systems.

Filtration is a typical method for removing particles from water, and it may also be used to remove micro plastics . The filtration process becomes increasingly difficult and costly as the particle size decreases. As a result, methods like membrane filtering have drawbacks including high investment costs, high energy consumption, and significant maintenance, for example, due to membrane scaling and fouling. Dissolved air flotation (D A F)  might be a simpler approach. Micro plastic removal efficiencies were found to be unsatisfactory in several investigations .

. Even when  surface modifiers were used, Wang et al. could only achieve micro plastic removal rates of 68.9\% to 43.8 percent with D A F . Micro plastics can be made up of a wide variety of polymers with varying characteristics and surface chemistries. These factors can have a significant impact on the interaction of flocculants with micro plastics, making it even more difficult to locate acceptable flo cculants .Because most regularly used flo cculants are made of iron or aluminium, their flexibility is restricted .Poly electrolyte-based flocculants are more flexible, but their solubility means they stay in the water and affect aquatic creatures and ecosystems .

Faced with this problem, Her bort et al. 2016 devised an organosilanes-based method for removing microplastics from water . One organic group and three reactive groups make up organosilanes. The organosilanes attach to the surface of the microplastics and collect it in agglomerates in the first step of the fixation process due to the interaction of the organic group and the surface of the microplastics. The three reactive groups produce a solid hybrid silica gel in the second phase of the fixing, which incorporates and fixes the microplastics chemically via a water induced sol gel process. The reactive groups are hydrolyzed to highly reactive silanols during the sol gel process, which then condense and form siloxane linkages 


Microplastics are removed from water using organosilanes in an agglomeration fixation reaction. In a water driven sol gel process, organosilanes bind to the surface of microplastics, aggregate it in huge agglomerates, and chemically fix it by generating a solid hybrid silica. Microplastic removal from water using organosilanes via agglomeration-fixation process. In a water-induced sol-gel process, organosilanes bind to the surface of microplastics, gather it in huge agglomerates, and chemically fix it by generating a solid hybrid silica. The combination of a physical agglomeration process and a water-induced chemical fixation process, which results in significant particle growth and stable agglomerates ,is what makes this approach unique.
As a result, different polymer types and surface chemistries may be tailored to the organic groups. Changes in reactive groups and organic groups can modify the reactivity of organosilanes to different water compositions. Organosilanes as a chemical substance class have a wide range of applications and flexibility, making this relatively new and understudied technique ideal for water treatment and microplastic removal. The procedure has been evaluated on a lab scale with demineralized water and on a pilot plant size with tap water to remove polyethylene and polypropylene based microplastics.
It was studied how water compositions impact the removal process, as dissolved ions, natural organic matter, and surfactant compounds might alter the sol-gel process and therefore the removal process [28], in order to study the transferability to processes in sea water and wastewater. The impact of water temperatures is also investigated, since they might differ based on climatic circumstances or wastewater source. Because the production of silanol groups improves water solubility, any leftover organosilane residues in the water after the fixation process are monitored.
No organosilanes may remain dissolved in the water after the procedure is completed to avoid potential discharge. The procedure was also tried for common polymer types with varying characteristics and surface chemistries to see how they impact the interaction with the organosilanes and the development of agglomerates. We want to get microplastic removal by organosilanes one step closer to application transfer by combining the examination of these novel and essential elements.

\clearpage
\section{Materials and Methods}
We evaluated the removal process in demineralized water, salt water, and municipal wastewater that had been biologically treated. The salt water was made by dissolving 3.5 percent by weight of untreated Atlantic sea salt in demineralized water (Art. No. 8530, Biova, Wildberg, Germany). On March 2, 2020, biologically processed wastewater was collected from the sewage treatment facility in Landau i. d. Pfalz, Germany (after several days of rain). Table S1 displays the water parameters. Water 2021, 13, 675 4 of 15 included was filtered via a 0.6 m paper filter (Macherey-Nagel MN 85/70 BF) to remove any particles.

\begin{figure}[h]
\centering
\includegraphics[scale=0.7]{GB2.jpg}
\caption{Plastics and Microplastic filter machine .}
\label{fig_ROB}
\end{figure}
The water samples were altered to 7.5, 20, and 40 degrees Celsius using ice baths or heating plates to compare the effects of different temperatures. 2.3. Removal Efficiency Determination According to Sturm et al. 2020 [27], the clearance efficiency was evaluated gravimetrically. 1 litre of water was poured into a 2 litre beaker, the microplastic was added, and the suspension was swirled for 5 minutes at the maximum speed using a magnetic stirrer. To execute the agglomeration process, the stirrer was adjusted to 500 rpm, the organosilane was added, and the mixture was swirled for 20 minutes. The contents of the beaker were filtered through an analytical sieve after 20 minutes .
Agglomerates bigger than 1 mm remained in the filter and were classed as eliminated as a result. To eliminate any associated organosilane residues, the filtrate with the residual free microplastic was continuously filtered through a filter crucible (porosity 4, max. pore size 16 m) and washed with isopropanol. The weight of the free microplastic may be calculated after drying the sample at 105°C for 24 hours. The organosilane dose was 100 L/L unless otherwise specified, and the microplastic concentration was 100 mg/L. When performing lab scale studies with 1 L water, the lowest concentration of microplastics per litre that could be used to generate consistent and repeatable findings was 100 mg microplastics per litre.
An organosilane dose of 100 L/L was chosen because, in earlier investigations, it was shown to be the lowest concentration that effectively removed 100 mg PE/PP from water (95\%) on a laboratory scale. The size restriction for agglomerates to be regarded as removed was chosen at 1 mm because in pilot plant studies, agglomerates of this size were easy to remove. Microplastics that adhere to the beaker's wall or bottom and cannot be removed by completely washing with the wash bottle are likewise deemed removed. 
\begin{figure}[h]
\centering
\includegraphics[scale=0.5]{GB3.jpg}
\caption{Plastics and Microplastic filter machine .}
\label{fig_ROB}
\end{figure}
. Organosilane Residues Dissolved in Water Determination The standardised removal procedure was carried out using 100 mg/L of a PE/PP combination (1:1) as microplastics to quantify the dissolved organosilane residues.
. A water sample was obtained after 20 minutes of churning and filtered using a 0.45 m syringe filter. An ICP-OES spectrometer (Inductive Coupled Plasma Optical Emission Spectroscopy) was used to determine the sample's silicon content . The Sievers TOC Analyzer 820 was used to determine the DOC content (dissolved organic carbon). The measured values were subtracted from the blank values from the various water samples. Analytical Support To estimate the mean size, 20 microplastic particles and 10 agglomerates were photographed and measured using a stereomicroscope digital camera. 
The aggregates and hybrid silica gels, which had been dried for 24 hours at 105°C, had their IR spectra (4000–300 cm1, resolution 1 cm1) recorded using the ATR-FTIR spectrometer Vertex 70, Bruker, Ettlingen, Germany. Thermogravimetric analysis (TGA) was performed using a platinum crucible and a Q5000 IR from TA Instruments. Starting temperature was 45 degrees Celsius, purge gas 1: nitrogen 5.0, 25 mL/min, and a heating rate of 20 degrees Celsius per minute. Purge gas 2: oxygen 5.0; 0.25 mL/min with a heating rate of 20 C/min and a final temperature of 950 C were used with a gas switching temperature of 600 C, purge gas 2: oxygen 5.0; 0.25 mL/min with a heating rate of 20 C/min and a final temperature of 950 C.

\clearpage
\section{conclusion}
This research found that the unique approach of microplastic removal using organosilanes has a lot of promise for application on a large scale. Temperature and water type, in this case salt water and biologically treated municipal wastewater, have no effect on microplastic removal with organosilanes, demonstrating the method's robustness and allowing it to be easily transferred to advanced wastewater treatment or sea water treatment processes. Because wastewater composition varies greatly depending on catchment region, weather, and climatic circumstances, more research is needed to guarantee that the method is applicable to a variety of wastewaters. In addition, using the method in very saline lakes might be difficult. 
The chemical composition and surface chemistry of microplastics were also shown to have a significant impact on the removal process and physical interaction with organosilanes. With increasing polarity of the polymer, the removal effectiveness of microplastics based on distinct polymer types diminishes. By increasing the polarity of the organic group, highly polar polymers may be extracted more efficiently. However, non-polar polymers are less effective as a result of this. These findings reveal that organosilanes may be tailored to optimise the removal of certain polymer types by matching the organic group to the polymer's surface chemistry. Organosilanes are a potential chemical class for this task because of their great diversity and modifiability.
Using larger concentrations of organosilanes is another way to improve efficiency. Further research should concentrate on the use of various
\begin{figure}[h]
\centering
\includegraphics[scale=0.7]{GB4.png}
\caption{Beautiful sea after filter from micro plastic .}
\label{fig_ROB}
\end{figure}

 organosilanes in combination to effectively remove combinations of polar and non-polar polymers found in the environment. It's also possible to examine the impact of biofilm coverage, weathering, and natural organic matter adsorption. For all polymer types, the size composition of microplastics and agglomerates shows a significant increase in size throughout the agglomeration process. However, on a pilot plant size, the 1 mm restriction for agglomerates for effective removal is frequently not met. On technological grounds, a better removal method for agglomerates with a lower size limit is projected to result in much improved microplastic removal.

After the removal procedure, the organosilanes n-butyltrichlorosilane and iso octyltrichlorosilane both show substantial amounts of residues dissolved in water, ranging from 76.4 to 46.3 percent. PE-X leaves no dissolved residues, making it ideal for use on a technical scale without the danger of introducing organosilanes into the environment or technical processes, which is a significant benefit of this particular organosilane. There are no significant variations in the quantity of residues found in the studied water compositions. ICP-OES provided more precise and dependable findings than DOC tests in controlling dissolved organosilanes residues. However, as a less expensive and speedier solution, DOC may still be used for effective process control in technical applications.
IR spectra, and TGA curves. The TGA curves can also reveal that mixes of various polymers can be fixed in agglomerates at the same time. As a result, these approaches are ideally adapted to investigating agglomeration composition and then controlling the process during continuous application. Because of the limits of the laboratory scale test equipment, 100 mg/L was chosen as the least microplastic concentration. Environmental concentrations, on the other hand, are normally much lower. As a result, future pilot-plant studies will concentrate on lowering microplastic concentrations as well as demonstrating functionality in real-world applications without microplastic spiking.
In addition, a wider range of water compositions will be examined. We believe that this approach has a strong potential to be a cost-effective and easy-to-apply alternative to typical flocculants due to its flexibility to numerous polymer types and little technical effort for use on a technical scale.




\end{document}