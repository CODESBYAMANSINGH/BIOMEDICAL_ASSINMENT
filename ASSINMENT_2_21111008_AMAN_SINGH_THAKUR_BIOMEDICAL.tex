\documentclass[12pt]{article}

\usepackage{graphicx}
\graphicspath{{img/}}
\usepackage{hyperref}
\hypersetup{colorlinks=true, citecolor=blue , linkcolor=blue,urlcolor=blue}

\title{NATIONAL INSTITUTE OF TECHNOLOGY\\
(NIT)RAIPUR (C.G.)}
\author {SESSION 2021-22}
\date{}

\usepackage{lipsum}
\usepackage{fancyhdr}
\pagestyle{fancy}
\lhead{ROLL NO. 21111008}
\rhead{PAGE \thepage}
\cfoot{AMAN SINGH THAKUR}
\renewcommand{\headrulewidth}{0.4pt}
\renewcommand{\footrulewidth}{0.4pt}
\begin{document}
\maketitle
\begin{figure}[h]
\centering
\includegraphics[scale=0.5]{NIT.png}
\end{figure}
\author {------------------------------ASSINMENT-2------------------------------ \\
.\> \>\> \>\> \>\>\> \>\> \>\>\>\>\>\>  \>\> \>\>\> \>\> \>\> \> \>\> \>\> \>\> \>\> ON\\ 
.\> \>\> \>\> \>\ {EVOLUTION OF THE MODERN HEALTH CARE SYSTEM} }\\


\author{GUIDANCE BY \> \>\> \>\> \>\>\> \>\> \>\>\>\>\>\>  \>\> \>\>\> \>\> \>\> \> \>\> \>\> \>\> \>\>  SUBMITTED BY }\\


\author{DR.SAURABH GUPTA\>  \>\> \>\> \>  \>\> \>\>\>\>\>  \>\> \>\>\>  \>\> \>\> \>  \>\> AMAN SINGH THAKUR }\\

\author{ ASSISTANT PROFESSOR\>  \>\> \>  \> \>   \>\> \>\> \> \>\> \> \>\> \>  ROLL NO.:--- 21111008 }\\
\author{BIOMEDICAL DEPARTMENT\>  \>\> \>\> \>  \>\> \>\> \>\>   \>\> BRANCHE:--BIOMEDICAL  }\\
\author{.\>   \>\> \>\> \>  \>\> \>\> \>\> \>\> \>\> \>  \>\> \>\> \>\>\>\> \>\> \>\>  \>\> \>\> \>\> \>\>\> \>\> \>   \>\> \>\> \>\> \>\>\> \>\> \>    \>\> \>\> \>\> \>\> \>\> \>  \>\> \>\> \>\>  \>\> ENGINEERING }
\clearpage
\tableofcontents
\clearpage

\section{EVOLUTION OF THE MODERN HEALTH CARE SYSTEM
}
Primitive humans considered diseases to be ''visitations,'' the whimsical acts of angry gods or spirits.
As a result, medical practice was the domain of the doctor and the medicine man and medicine woman.
Even as magic became an integral part of the healing process, the cult and the art of these early practitioners were never entirely limited to the supernatural.
These individuals, by using their natural instincts and learning from experience, developed a primitive science based on empirical laws.
Through acquisition and coding of certain reliable practices, the arts of herb doctoring, bone setting, and surgery  were advanced.
Just as primitive humans learned from observation that certain plants and grains were good to eat and could be cultivated, so the healers and shamans observed the nature of certain illnesses and then passed on their experiences to other generations.
Evidence indicates that the primitive healer took an active, rather than a simply intuitive interest in the curative arts, acting as a surgeon and a user of tools.
Skulls with holes made in them by trephiners have been collected in various parts of Europe, Asia, and South America.
These holes were cut out of the bone with flint instruments to gain access to the brain.
Perhaps this procedure liberated from the skull the malicious demons that were thought to be the cause of extreme pain or attacks of falling to the ground.
That this procedure was carried outon living patients, some of whom actually survived, is evident from the rounded edges on the bone surrounding the hole which indicate that the bone had grown again after the operation.
From these beginnings, the practice of medicine has become integral to all human societies and cultures.
It is interesting to note the fate of some of the most successful of these early practitioners.\\


\>  The Egyptians, for example, have held Imhotep, the architect of the first pyramid, in great esteem through the centuries, not as a pyramid 3 builder, but as a doctor.
'' This early physician practiced his art so well that he was deified in the Egyptian culture as the god of healing.
Egyptian mythology, like primitive religion, emphasized the interrelationships between the supernatural and one's health.
Consider the mystic sign Rx, which still adorns all prescriptions today.
It has a mythical origin in the legend of the Eye of Horus.
It appears that as a child Horus lost his vision after being viciously attacked by Seth, the demon of evil.
The mother of Horus, called for assistance to Thoth, the most important god of health, who promptly restored the eye and its powers.
Because of this intervention, the Eye of Horus became the Egyptian symbol of godly protection and recovery, and its descendant, Rx, serves as the most visible link between ancient and modern medicine.\\

The concepts and practices of Imhotep and the medical cult he fostered were duly recorded on papyri and stored in ancient tombs.
These writings outline proper diagnoses, prognoses, and treatment in a series of surgical cases.
These two papyri are certainly among the outstanding writings in medical history.
As the influence of ancient Egypt spread, Imhotep was identified by the Greeks with their own god of healing, Aesculapius.
According to legend, the god Apollo fathered Aesculapius during one of his many earthly visits.
Apparently Apollo was a concerned parent, and, as is the case for many modern parents, he wanted his son to be a physician.
He made Chiron, the centaur, tutor Aesculapius in the ways of healing.
Chiron's student became so proficient as a healer that he soon surpassed his tutor and kept people so healthy that he began to decrease the population of Hades.
Pluto, the god of the underworld, complained so violently about this course of events that Zeus killed Aesculapius with a thunderbolt and in the process promoted Aesculapius to Olympus as a god.
Inevitably, mythology has become entangled with historical facts, and it is not certain whether Aesculapius was in fact an earthly physician like Imhotep, the Egyptian.
One thing is clear; by 1000 BC, medicine was already a highly respected profession.
In Greece, the Aesculapia were temples of the healing cult and may be considered among the first hospitals.
In modern terms, these temples were essentially sanatoriums that had strong religious overtones.\\

\begin{figure}[h]
\centering
\includegraphics[scale=0.5]{AS2.jpg}
\caption{Illustration of sick child brought into the temple of Aesculapius}
\label{fig_AS2}
\end{figure}

Were received and psychologically prepared, through prayer and sacrifice, to appreciate the past achievements of Aesculapius and his physician priests.
.[FIGURE\ref{fig_AS2}]
'' During the night, ''healers'' visited their patients, administering medical advice to clients who were awake or interpreting dreams of those who had slept.
With this approach, patients, not treatments, were at fault if they did not get well.
The notion of ''healthy mind, healthy body'' is still in vogue today.
One of the most celebrated of these ''healing'' temples was on the island of Cos, the birthplace of Hippocrates, who as a youth became acquainted with the curative arts through his father, also a physician.
Hippocrates was not so much an 5 innovative physician as a collector of all the remedies and techniques that existed up to that time.
Since he viewed the physician as a scientist instead of a priest, Hippocrates also injected an essential ingredient into medicine: its scientific spirit.
Instead of blaming disease on the gods, Hippocrates taught that disease was a natural process, one that developed in logical steps, and that symptoms were reactions of the body to disease.
The body itself, he emphasized, possessed its own means of recovery, and the function of the physician was to aid these natural forces.
Hippocrates treated each patient as an original case to be studied and documented.
His shrewd descriptions of diseases are models for physicians even today.
Hippocrates and the school of Cos trained a number of individuals who then migrated to the corners of the Mediterranean world to practice medicine and spread the philosophies of their preceptor.
The work of Hippocrates and the school and tradition that stem from him constitute the first real break from magic and mysticism and the foundation of the rational art of medicine.
As a practitioner, Hippocrates represented the spirit, not the science, of medicine, embodying the good physician: the friend of the patient and the humane expert.\\
   
   As the Roman Empire reached its zenith and its influence expanded across half the world, it became heir to the great cultures it absorbed, including their medical advances.
Although the Romans themselves did little to advance clinical medicine , they did make outstanding contributions to public health.
They had a well-organized army medical service, which not only accompanied the legions on their various campaigns to provide ''first aid'' on the battlefield but also established ''base hospitals'' for convalescents at strategic points throughout the empire.
The construction of sewer systems and aqueducts were truly remarkable Roman accomplishments that provided their empire with the medical and social advantages of sanitary living.
Insistence on clean drinking water and unadulterated foods affected the control and prevention of epidemics, and however primitive, made urban existence possible.
Without adequate scientific knowledge about diseases, all the preoccupation of the Romans with public health could not avert the periodic medical disasters, particularly the plague, that mercilessly befell its citizens.
\clearpage
\section{ENGINEERING IN MODERN MEDICINE}
Modern medical practice actually began at the turn of the twentieth century.
Medicine had little to offer the average citizen since its resources were mainly physicians, their education, and their little black bags.
At this time physicians were in short supply, but for different reasons than exist today.
Only in the twentieth century did the tremendous explosion in scientific knowledge and technology lead to the development of the American health care system with the hospital as its focal point and the specialist physician and nurse as its most visible operatives.\\


In the twentieth century, advances in the basic sciences began to occur much more rapidly.
It was an era of intense interdisciplinary cross-fertilization.
Discoveries in the physical sciences enabled medical researchers to take giant strides forward.
In 1903 William Einthoven devised the first electrocardiograph and measured the electrical changes that occurred during the beating of the heart.
In the process, Einthoven initiated a new age for both cardiovascular medicine and electrical measurement techniques.\\



Of all the new discoveries that followed one another like intermediates in a chain reaction, the most significant for clinical medicine was the development of x-rays.
When W.K. Roentgen described his ''new kinds of rays,'' the human body was opened to medical inspection.
Initially these x-rays[FIGURE \ref{fig_XR2}] were used in the diagnosis of bone fractures and dislocations.
In the United States, x-ray machines brought this modern technology to most urban hospitals.
In the process, separate departments of radiology were established, and the influence of their activities spread, with almost every department of medicine advancing with the aid of this new tool.
By the 1930s, x-ray visualization of practically all the organ systems of the body was possible by the use of barium salts and a wide variety of radiopaque materials.\\
\begin{figure}[h]
\centering
\includegraphics[scale=0.5]{XR2.jpg}
\caption{Photograph of a modern medical imaging facility (X-ray MACHINE)
}
\label{fig_XR2}
\end{figure}


The power this technological innovation gave physicians was enormous.
The x\- ray permitted them to diagnose a wide variety of diseases and injuries accurately.
Being within the hospital, it helped trigger the transformation of the hospital from a passive receptacle for the sick poor to an active curative institution for all citizens of the American society .\\


The introduction of sulfanilamide in the mid-1930s and penicillin in the early 1940s significantly reduced the main danger of hospitalization: cross infection among patients.
With these new drugs in their arsenals, surgeons were able to perform their operations without prohibitive morbidity and mortality due to infection.
Until that time, ''fresh'' donors were bled, and the blood was transfused while it was still warm.\\


As technology in the United States blossomed so did the prestige of American medicine.
From 1900 to 1929 Nobel Prize winners in physiology or medicine came primarily from Europe, with no American among them.
In the period 1930 to 1944, just before the end of World War II, seven Americans were honored with this award.
During the post-war period of 1945 to 1975, 37 American life scientists earned similar honors, and from 1975-2003, the number was 40.
Thus, since 1930 a total of 79 American scientists have performed research significant enough to warrant the distinction of a Nobel Prize.
Most of these efforts were made possible by the technology available to these clinical scientists[FIGURE\ref{fig_MR1}].\\

\begin{figure}[h]
\centering
\includegraphics[scale=2.8]{MR1.png}
\caption{Early electrocardiograph machine
}
\label{fig_MR1}
\end{figure}
The employment of the available technology assisted in advancing the development of complex surgical procedures.
In the 1940s, cardiac catheterization and angiography were developed.
Accurate diagnoses of congenital and acquired heart disease also became possible, and a new era of cardiac and vascular surgery began.
Another child of this modern technology, the electron microscope, entered the medical scene in the 1950s and provided significant advances in visualizing relatively small cells.
Body scanners to detect tumors arose from the same science that brought societies reluctantly into the atomic age.
These ''tumor detectives'' used radioactive material and became commonplace in newly established departments of nuclear medicine in all hospitals.
The health care system that consisted primarily of the ''horse and buggy'' physician was gone forever, replaced by the doctor backed by and centered around the hospital, as medicine began to change to accommodate the new technology.\\


Following World War II, the evolution of comprehensive care greatly accelerated.
The advanced technology that had been developed in the pursuit of military objectives now became available for peaceful applications with the medical profession benefiting greatly from this rapid surge of technological finds.
The realm of electronics came into prominence.
Science and technology have leap-frogged past one another throughout recorded history.
Anyone seeking a causal relation between the two was just as likely to find technology the cause and science the effect as to find science the cause and technology the effect.
With the advent of electronics this causal relation between technology and science changed to a systematic exploitation of scientific research and the pursuit of knowledge that was undertaken with technical uses in mind.
The list becomes endless when one reflects upon the devices produced by the same technology that permitted humans to stand on the moon.
What was considered science fiction in the 1930s and the 1940s became reality.
Devices continually changed to incorporate the latest innovations, which in many cases became outmoded in a very short period of time.
Telemetry devices used to monitor the activity of a patient's heart freed both the physician and the patient from the wires that previously restricted them to the four walls of the hospital room[FIGURER\ref{fig_XR2}].\\


Similar to those that controlled the flight plans of the Apollo capsules, now completely inundate our society.
Since the 1970s, medical researchers have put these electronic brains to work performing complex calculations, keeping records, and even controlling the very instrumentation that sustains life.
The development of new medical imaging techniques such as computerized tomography and magnetic resonance imaging totally depended on a continually advancing computer technology.\\

Technology to provide prosthetic devices such as artificial heart valves and artificial blood vessels developed.
Even an artificial heart program to develop a replacement for a defective or diseased human heart began.
These technological innovations radically altered surgical organization and utilization.
The comparison of a hospital in which surgery was a relatively minor activity as it was a century ago to the contemporary hospital in which surgery plays a prominent role dramatically suggests the manner in which this technological effort has revolutionized the health profession and the institution of the hospital.\\


Through this evolutionary process, the hospital became the central institution that provided medical care.
Because of the complex and expensive technology that could be based only in the hospital and the education of doctors oriented both as clinicians and investigators toward highly technological norms, both the patient And the physician were pushed even closer to this center of attraction.
The effects of the increasing maldistribution and apparent shortage of physicians during the 1950s and 1960s also forced the patient and the physician to turn increasingly to the ambulatory clinic and the emergency ward of the urban hospital in time of need.
\begin{figure}[h]
\centering
\includegraphics[scale=0.5]{MR.png}
\caption{Photograph of a modern medical imaging facility (MRI)
}
\label{fig_MR}
\end{figure}
Emergency wards today handle not only an ever-increasing number of accidents and somatic crises such as heart attacks and strokes, but also problems resulting from the social environments that surround the local hospital.[FIGURE\ref{fig_MR}]
Added to these individuals are those who live in the neighborhood of the hospital and simply cannot afford their own physician.
Often such individuals enter the emergency ward for routine care of colds, hangovers, and even marital problems.
Because of these developments, the hospital has evolved as the focal point of the present system of health care delivery.
The hospital, as presently organized, specializes in highly technical and complex medical procedures.
This evolutionary process became inevitable as technology produced increasingly sophisticated equipment that private practitioners or even large group practices were economically unequipped to acquire and maintain.\\

Only the hospital could provide this type of service.
The steady expansion of scientific and technological innovations has not only necessitated specialization for all health professionals but has also required the housing of advanced technology within the walls of the modern hospital.



\end{document}