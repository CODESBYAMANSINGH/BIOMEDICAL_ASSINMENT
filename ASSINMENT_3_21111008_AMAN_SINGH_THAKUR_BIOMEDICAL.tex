\documentclass[12pt]{article}

\usepackage{graphicx}
\graphicspath{{img/}}
\usepackage{hyperref}
\hypersetup{colorlinks=true, citecolor=blue , linkcolor=blue,urlcolor=blue}

\title{NATIONAL INSTITUTE OF TECHNOLOGY\\
(NIT)RAIPUR (C.G.)}
\author {SESSION 2021-22}
\date{}

\usepackage{lipsum}
\usepackage{fancyhdr}
\pagestyle{fancy}
\lhead{ROLL NO. 21111008}
\rhead{PAGE \thepage}
\cfoot{AMAN SINGH THAKUR}
\renewcommand{\headrulewidth}{0.4pt}
\renewcommand{\footrulewidth}{0.4pt}
\begin{document}
\maketitle
\begin{figure}[h]
\centering
\includegraphics[scale=0.5]{NIT.png}
\end{figure}
\author {--------------ASSINMENT-3 ON {FUTURE OF HEALTHCARE} --------------}\\


\author{GUIDANCE BY \> \>\> \>\> \>\>\> \>\> \>\>\>\>\>\>  \>\> \>\>\> \>\> \>\> \> \>\> \>\> \>\> \>\>  SUBMITTED BY }\\


\author{DR.SAURABH GUPTA\>  \>\> \>\> \>  \>\> \>\>\>\>\>  \>\> \>\>\>  \>\> \>\> \>  \>\> AMAN SINGH THAKUR }\\

\author{ ASSISTANT PROFESSOR\>  \>\> \>  \> \>   \>\> \>\> \> \>\> \> \>\> \>  ROLL NO.:--- 21111008 }\\
\author{BIOMEDICAL DEPARTMENT\>  \>\> \>\> \>  \>\> \>\> \>\>   \>\> BRANCHE:--BIOMEDICAL  }\\
\author{.\>   \>\> \>\> \>  \>\> \>\> \>\> \>\> \>\> \>  \>\> \>\> \>\>\>\> \>\> \>\>  \>\> \>\> \>\> \>\>\> \>\> \>   \>\> \>\> \>\> \>\>\> \>\> \>    \>\> \>\> \>\> \>\> \>\> \>  \>\> \>\> \>\>  \>\> ENGINEERING }
\clearpage
\tableofcontents
\clearpage

\section{FUTURE OF HEALTHCARE}
Each industry has its owns challenges, and digital disruption is everywhere.
Constant change requires scenario-based thinking, exploring several paths and crafting a digital strategy based on preparing for the future.
For medical field to stay relevant, they need to explore the future and look at next generations.\\




\subsection{Next-level patient care}
More than ever before, people are taking an active role in their own health.
Well-informed patients are making healthier lifestyle choices and adjusting their diets.
Their relationship with healthcare providers is still reactive.
Healthcare professionals struggle to take an accurate history of the patient and make their best estimation of the potential causes of the problem.
With the ambition to live longer, more vibrant, healthier lives, the patients of the future will want to do more than solve problems when they occur.
They'll want to address potential issues before they become problems, and stop genetic deficiencies in their tracks.
Achieving these goals will require a proactive approach that includes using technology to take preventative action and achieve long-term insight.\\

\> With the wealth of data collected, healthcare professionals can monitor patients and ensure that issues are identified long before they have a noticeable impact on health and well-being.
The tech-savvy and health-conscious patients of tomorrow will welcome these new devices into their lives.
What better way to avoid the difficulties of long-term illness than to continuously monitor the body and receive alerts at the earliest signs of trouble. Patients will also form a new relationship with their physicians.
Instead of dreading the bad news that often comes from a doctor's visit, patients will welcome the insights that will enable them to take early action on potential health issues.\\
 
\clearpage
\subsection{Using Technologies in HEALTHCARE} Technologies like telemedicine and augmented/virtual reality 
(AR/VR).
 [FIGURE \ref{fig_VR}] will transform the patient experience and reduce the 
fear and anxiety of medical visits. Patients will be constantly 
aware of their health status, and will be in more regular contact 
with the professionals who can guide them[FIGURE \ref{fig_VR}]. Blockchain will keep 
patient records secure and transparent, and the Internet of 
Things (IoT) will provide all the devices and gadgets that make 
monitoring automatic and hassle-free. 
Machine-learning technologies will allow patients to track their 
entire health history, recognize patterns and get alerts when 
something goes off track.
\begin{figure}[h]
\centering
\includegraphics[scale=0.3]{VR.jpg}
\caption{Using virtual reality  Technologies in healthcare.}
\label{fig_VR}
\end{figure}

\subsection{Technology that transforms}
Other advanced technologies will transform human health as we know it, from personalized medicines that are programmed for each patient's specific needs, metabolism and lifestyle, to babies born free of any preventable diseases.
Many companies are already addressing the need for viable human organs, using technology that produces biocompatible organs on a 3D printer and  AI to diagnose, monitor [FIGURE \ref{fig_ALR}] and test solutions for those suffering from epilepsy
 \begin{figure}[h]
\centering
\includegraphics[scale=0.3]{ALR.png}
\caption{Monitoring  human organ system with help of advanced technologies.}
\label{fig_ALR}
\end{figure}
.\\


 \subsection{The future of healthcare is human} The patient of 2025 will be better informed, more health conscious and more self-motivated than any time in the past.
With the possibility of living well past 100, patients will reach out to the technologies that take the fear and uncertainty out of healthcare, and replace it with monitoring, education and support.\\

 \> By embracing Internet of Things IoT, AI and machine learning, healthcare companies can become active participants in patients' long-term well being, instead of the bearers of bad news when illness remains undetected for too long.
Chronic illnesses like obesity, diabetes and high blood pressure will cease to be life-long struggles as real-time insights enable patients and doctors to work together to resolve issues.
Rather than waiting until symptoms are impossible to ignore, patients will heed the earliest warning signs and make better health decisions.
\clearpage
\subsection{Algorithmic healthcare}
With increasingly intelligent algorithms, smart machines are becoming able to make decisions about clinical treatments, medicines and diagnoses.
 Developments in artificial intelligence like deep learning and machine learning are also helping the medical world move from descriptive to predictive and even prescriptive care.[FIGURE \ref{fig_MD}]
Not only can algorithms diagnose diseases like cancer and cardiovascular illnesses, but they can also predict a mental breakdown or depression by analyzing our voice.
\begin{figure}[h]
\centering
\includegraphics[scale=0.6]{MD.png}
\caption{Algorithmic healthcare.}
\label{fig_MD}
\end{figure}\\
Beyond a diagnosis, algorithms can also make predictions like length of hospital stay, chances of getting out of a coma or a patient's odds of dying.
While most of these predictions still come from analyzing patterns in existing data, we'll eventually enter a world where algorithms will answer patients directly on what they should do before getting ill.
The rise of algorithms makes it possible for healthcare professionals to focus more on patient care.

Rather than replacing them, it will improve patient care by making it more personalized, proactive and efficient.

Doctors are likely to embrace smart algorithms because these medical professionals tend to favor using new technologies to improve their healthcare services.
\clearpage
\subsection{Merge of technology with the human body}
While advanced technologies make all this possible, connected 
healthcare also starts with achieving digital health literacy, 
including the ability to think critically about healthcare 
information sources, services, products and apps – a challenge 
given the onslaught of information available today. Doctors can 
play a role in helping patients connect, with the rise of smart and 
autonomous algorithms, as well as other new technologies that 
help personalize information and services in healthcare.\\

\> Technology is increasingly seen as a way to augment and even remake ourselves toward higher ideals.
This idea isn't completely new - from wooden legs to glasses and hearing aids, humans have tried to improve their biological bodies.
We want to fight the existing limitations of aging with new and improved technologies.\\
LIKE--
\begin{figure}[h]
\centering
\includegraphics[scale=0.3]{ALR3.png}
\caption{Doctor's monitoring  human organ system with help of advanced technologies.}
\label{fig_ALR3}
\end{figure}

 Health monitoring [FIGURE \ref{fig_ALR3}]  and support system that 
operates in the homes of elderly people. Companies are
developed a smart algorithm, Addison, that continuously monitors 
people, and asks basic questions about their health and well-being. 
Addison can remind users to follow through on nutrition plans, 
take their medication, check their vitals, connect them 24/7 with a 
medical professional and provide assistance during emergencies [FIGURE \ref{fig_ALR1}].
\begin{figure}[h]
\centering
\includegraphics[scale=0.3]{ALR1.png}
\caption{Patient interact with Doctor using technology.}
\label{fig_ALR1}
\end{figure}\\ 


\end{document}